% \iffalse meta-comment
% !TEX program  = pdfLaTeX
%<*internal>
\iffalse
%</internal>
%<*readme>
--------------------------------------------------------------------
ncsuthesis --- A thesis class conforming to NC State ETD guidelines.
E-mail: jmhykes@ncsu.edu
Released under the LaTeX Project Public License v1.3c or later
See http://www.latex-project.org/lppl.txt
--------------------------------------------------------------------

The ncsuthesis class provides formatting, macros, and content
to meet the thesis requirements of North Carolina State University,
specifically the ETD guidelines.
%</readme>
%<*internal>
\fi
\def\nameofplainTeX{plain}
\ifx\fmtname\nameofplainTeX\else
  \expandafter\begingroup
\fi
%</internal>
%<*install>
\input docstrip.tex
\keepsilent
\askforoverwritefalse
\preamble
--------------------------------------------------------------------
ncsuthesis --- A thesis class conforming to NC State ETD guidelines.
E-mail: jmhykes@ncsu.edu
Released under the LaTeX Project Public License v1.3c or later
See http://www.latex-project.org/lppl.txt
--------------------------------------------------------------------

\endpreamble
\postamble

Copyright (C) 2009-2010 by Josh Hykes <jmhykes@ncsu.edu>

This work may be distributed and/or modified under the
conditions of the LaTeX Project Public License (LPPL), either
version 1.3c of this license or (at your option) any later
version.  The latest version of this license is in the file:

http://www.latex-project.org/lppl.txt

This work is "maintained" (as per LPPL maintenance status).

This work consists of the file  ncsuthesis.dtx
and the derived files           ncsuthesis.ins,
                                ncsuthesis.pdf and
                                ncsuthesis.cls.

\endpostamble
\usedir{tex/latex/ncsuthesis}
\generate{
  \file{\jobname.cls}{\from{\jobname.dtx}{class}}
}
%</install>
%<install>\endbatchfile
%<*internal>
\usedir{source/latex/ncsuthesis}
\generate{
  \file{\jobname.ins}{\from{\jobname.dtx}{install}}
}
\nopreamble\nopostamble
\usedir{doc/latex/ncsuthesis}
\generate{
  \file{README.txt}{\from{\jobname.dtx}{readme}}
}
\ifx\fmtname\nameofplainTeX
  \expandafter\endbatchfile
\else
  \expandafter\endgroup
\fi
%</internal>
%<*class>
\NeedsTeXFormat{LaTeX2e}
\ProvidesClass{ncsuthesis}[2010/11/02 v0.2.3 NC State ETD conformant thesis class]
%</class>
%<*driver>
\documentclass{ltxdoc}
\usepackage[T1]{fontenc}
\usepackage[adobe-utopia]{mathdesign}
\usepackage[scaled]{berasans}
\usepackage[numbered]{hypdoc}
\EnableCrossrefs
\CodelineIndex
\RecordChanges
\OnlyDescription
\usepackage{xcolor}
\usepackage{sectsty}
\usepackage{microtype}
\newcommand{\themecolor}{\color{blue!55!black}}
\allsectionsfont{\themecolor\sffamily}
\renewcommand{\abstractname}{\themecolor\textsf{Abstract}}

\newcommand{\tc}{\textsf{ncsuthesis}} % the class
\begin{document}
  \DocInput{\jobname.dtx}
\end{document}
%</driver>
% \fi
% 
%\GetFileInfo{\jobname.cls}
%
%\title{
%  {\tc{} --- \textsf{NC State Thesis}}
% }
%
%\author{
%  Joshua Hykes\thanks{E-mail: jmhykes@ncsu.edu}
%}
%\date{Released 4/10/2010}
%
%\maketitle
%
%\begin{abstract}
%  The \tc{} class provides formatting, macros, and content
%  to meet the thesis requirements of North Carolina State University,
%  specifically the ETD guidelines.
%  
%\end{abstract}
% 
%\section*{Introduction}
%  Composing the content for a dissertation or thesis is a daunting task.  
%  The goal of the \tc{} class is to let the author focus on the content
%  while the formatting is handled transparently, in the tradition of  
%  \LaTeX{}.
%
%\section*{Design Choices}
%  The NC State guidelines are given in a
%  \href{http://www.ncsu.edu/grad/etd/docs/etd-guide.pdf}{PDF document}
%  at the 
%  \href{http://www.ncsu.edu/grad/etd/}{ETD website}.
%  Although parts of the formatting rules are strict, there is still
%  considerable flexibility in formatting the thesis document.  A few examples
%  where choices exist follow.
%  \begin{description}
%    \item[Font size] Sizes 10 to 12 point is acceptable, although 10 point
%                     is probably too small.  Consider using 11 or 12 point.
%    \item[Type face] Most \TeX{} and \LaTeX{} users use the default
%                     Computer Modern fonts.  While this type is perfectly
%                     acceptable,
%       \href{http://www.tug.dk/FontCatalogue/}{other options}
%                     are available that might better suit your taste.
%    \item[Heading style] There are lots of headings and section titles
%                     in the document, at the top of the \emph{Abstract},
%                     \emph{Dedication}, and \emph{References} pages, for 
%                     instance.  There are no requirements
%                     for these headings, except that that the style be
%                     uniform.  The \tc{} class provides commands that
%                     control the styles so that making changes is simple.
%                     One common typographical style is to set headings
%                     in a \textsf{sans serif font}.
%
%    \item[Margin] The margin has been set to the minimum required.
%                     Feel free to increase the margins in the class
%                     file, as the default line lengths are a bit long.
%                     For example, using Computer Modern at 11 point, the default
%                     margins give line lengths of about 90 characters,
%                     while experts typically recommend 60 to 70
%                     characters.
%
%    \item[Line spacing] ETD rules permit either 1.5 or double spacing.
%                     One and a half spacing will give a look closer to
%                     a normal book, but other considerations may tip the
%                     scales for double spacing.  It is easy to change between
%                     the two using \tc{}, passing either the 
%                     \textsf{onehalfspacing} or \textsf{doublespacing}
%                     options in the \verb^\documentclass^ declaration.
%
%    \item[Page number location] ETD requirements offer four options:
%                     upper right, upper center, lower right, and lower
%                     center.  The \tc{} class places the page number
%                     at the bottom center of the page, and does not 
%                     offer options to easily switch.  (It is possible 
%                     to place it at one of the other locations, but one would
%                     need to modify the \textsf{ncsuthesis.cls} source.)
%
%                     The bottom center was selected for the following reasons.
%                     First, the center seems more natural since this is
%                     a one-sided document.  Second, the bottom was chosen
%                     so that the bottom margin would be larger than the 
%                     top margin.  The guidelines require that the page number be
%                     one inch from the edge of the paper, and that the rest
%                     of the text be two double-spaced lines from the page 
%                     number.  Normally, the page number is set in the margins,
%                     but in this case, the rules effectively increase the
%                     top (for numbering at the top) or bottom (for
%                     numbering at the bottom) margin. Given that either
%                     the top or the bottom margin would be larger, the
%                     bottom seemed a better choice since 
%        \href{http://www.ctan.org/tex-archive/macros/latex/contrib/koma-script/scrguien.pdf}
%                     {page layouts commonly} use a larger bottom margin (also see
%        \href{http://books.google.com/books?id=940sAAAAYAAJ&dq=elements+of+typographic+style}
%                     {\textit{The Elements of Typographic Style}} by Robert Bringhurst or
%        \href{http://books.google.com/books?id=l-ZdxgKghVMC&lpg=PT45&ots=PsEcEOAKHj&dq=Hans%20Peter%20Willberg%20und%20Friedrich%20Forssman&pg=PT116#v=onepage&q=&f=false}
%                     {\textit{Getting it Right with Type}} by Victoria Squire),
%                     whereas a larger top margin is uncommon.
%                     
%  \end{description}
%
%\section*{Acknowledgements}
%  The \tc{} class was written from scratch, as much as is possible in our
%  internet age.  Peter Bloomfield made a number of helpful suggestions and
%  improvements, especially regarding the page headings.  
%  \href{http://www4.stat.ncsu.edu/~bloomfld/latex.html}{His \LaTeX\ website} 
%  has a number of helpful resources.
%  The \href{http://www.esm.psu.edu/psuthesis/}{Penn State 
%  \LaTeX{} thesis class} written by Professors Gary 
%  Gray and Francesco Costanzo was used as a reference for many of the 
%  \LaTeX{} programming constructs.  The previous NCSU thesis class,
%  from which some of the formatting rules were extracted,
%  had additions made by Eric Adles in March 2007, Jason Brinkley in
%  August 2008, Arnav Jhala in December 2008, Manish Sachdev in
%  January 2008, and Terry Byron.
%  
%\changes{v0.2}{2010/01/19}{A number of changes suggested by Dr.~Bloomfield, 
%             including improved chapter headings, checking for a period
%             after a middle initial, and natbib compatibility.}
%
%\changes{v0.2.1}{2010/02/24}{Appendix spacing improvements and a few other
%             patches from Dr.~Bloomfield.}
%
%\changes{v0.2.2}{2010/04/10}{Support for up to six committee members, and
%             the ability to change the titles listed under each member.
%             A patch from Dr.~Bloomfield to set URL hyperlinks in black,
%             as well as a related suggestion from Qiang Wang about another
%             hyperlink color issue.}
%
%\changes{v0.2.3}{2010/11/02}{Changed the method of adding space between 
%             colon and caption text in figures and tables.  The previous
%             method caused the space to appear in List of Figures and Tables.
%             Increased slightly the spacing between bottom of text and page 
%             number since the thesis editor complained about this to one
%             student. Added the textcomp package to get a better copyright 
%             sign.}
%
%\changes{v0.2.4}{2010/11/18}{Added the \textsf{letterpaper} option in case
%             someone's \TeX\ installation default isn't letterpaper.}
%
%\changes{v0.3}{2011/04/12}{Handling co-chairs correctly by Dr.~Bloomfield.
%             Also uploaded to Github.}
%
%
%\StopEventually{
%  \clearpage
%  \PrintChanges
%  %\PrintIndex
%}
%
%\section*{The Code}
%<*class>
%% -------------------------------------------------------------------------- %%
%% ------------------------------ Thesis type ------------------------------- %%
%% -------------------------------------------------------------------------- %%
\newcommand{\ncsu@thesistype}{phd}
\DeclareOption{phd}{\renewcommand{\ncsu@thesistype}{phd}}
\DeclareOption{ms}{\renewcommand{\ncsu@thesistype}{ms}}

%% Spacing options
\newcommand{\ncsu@defaultspacing}{\onehalfspacing}
\DeclareOption{singlespacing}{
     \renewcommand{\ncsu@defaultspacing}{\singlespacing}}
\DeclareOption{onehalfspacing}{
     \renewcommand{\ncsu@defaultspacing}{\onehalfspacing}}
\DeclareOption{doublespacing}{
     \renewcommand{\ncsu@defaultspacing}{\doublespacing}}

\newenvironment{lscape}{\begin{lscapenum}}{\end{lscapenum}}
\DeclareOption{hardcopy}{
    \renewenvironment{lscape}{\begin{lscapenotnum}}{\end{lscapenotnum}}
}

\DeclareOption{draft}{\PassOptionsToClass{\CurrentOption}{book}}
\DeclareOption{10pt}{\PassOptionsToClass{\CurrentOption}{book}}
\DeclareOption{11pt}{\PassOptionsToClass{\CurrentOption}{book}}
\DeclareOption{12pt}{\PassOptionsToClass{\CurrentOption}{book}}

\DeclareOption*{\PackageWarning{ncsuthesis}{Unknown option `\CurrentOption'. Ignoring}}
\ExecuteOptions{phd,onehalfspacing} % the default option is <phd, onehalfspacing>
\ProcessOptions
\LoadClass[openany,oneside,titlepage,letterpaper]{book}

\RequirePackage{setspace}
\RequirePackage{hyphenat}
\RequirePackage{nolbreaks}
\RequirePackage{ifthen}
\RequirePackage{sectsty}
\RequirePackage[titles]{tocloft}
\RequirePackage{graphicx}
\RequirePackage[nottoc]{tocbibind} % includes lof, lot, refs in toc
\RequirePackage{calc}

%%----------------------------------------------------------------------------%%
%%---------------------------- Formatting Options ----------------------------%%
%% -------------------------------------------------------------------------- %%
%% Capital letter height
\newlength{\ncsu@capitalheight}
\setlength{\ncsu@capitalheight}{1.6ex}
\settoheight{\ncsu@capitalheight}{D}

%% Line height
\newlength{\ncsu@lineheight}
\setlength{\ncsu@lineheight}{2.0ex}

%% We must add two double spaces plus one capital-height (for the page number)
%%  to footskip.  Footskip is measured from the top of the bottom margin.
%%  (See geometry package documentation, note that includefoot option is used.)
\ncsu@defaultspacing
\newlength{\ncsu@footskip}
\setlength{\ncsu@footskip}{\ncsu@capitalheight}
\addtolength{\ncsu@footskip}{2\baselineskip}
\addtolength{\ncsu@footskip}{0.5ex}  % A little extra space to ensure there
                                     %  are 2 full double spaced lines
\RequirePackage[includefoot,
                margin=1in,
                lmargin=1.25in,
                footskip=\ncsu@footskip]{geometry}

%% -------------------------------------------------------------------------- %%
%% Disposition format -- any titles, headings, section titles
\newcommand{\ncsu@dispositionformat}{}
\allsectionsfont{\ncsu@dispositionformat}
\providecommand{\dispositionformat}[1]{
    \renewcommand{\ncsu@dispositionformat}{#1}
    \allsectionsfont{\ncsu@dispositionformat}
}

%% Heading/Chapter/Section format
\newcommand{\ncsu@headingformat}{\ncsu@dispositionformat}
\providecommand{\headingformat}[1]{
    \renewcommand{\ncsu@headingformat}{\ncsu@dispositionformat #1}
}

%% -------------------------------------------------------------------------- %%
%% Headers/footers - page number
\pagestyle{plain}

%% -------------------------------------------------------------------------- %%
%% Vertical space around tables and figures, triple spacing
\setlength{\textfloatsep}{3\ncsu@lineheight}
\setlength{\intextsep}{3\ncsu@lineheight}

%%----------------------------------------------------------------------------%%
%%---------------------------  Content Options -------------------------------%%
%% -------------------------------------------------------------------------- %%
%% Author name
\newcommand{\ncsu@authorfirstmiddle}{First Middle}
\newcommand{\ncsu@authorlast}{Last}

%% Command to test for period after middle name
\newboolean{@ncsu@needsperiod}
\newcommand{\ncsu@testlastchar}[2]{%
    \ifthenelse{\equal{#2}{\relax}}{
        \ifthenelse{\equal{#1}{.}}{%
            \setboolean{@ncsu@needsperiod}{false}
        }{% #1 != .
            \setboolean{@ncsu@needsperiod}{true}
        }
        \let\next=\relax
    }
    {% #2 != relax
        \let\next=\ncsu@testlastchar
    }
    \next#2
}

\providecommand{\student}[2]{
    \renewcommand{\ncsu@authorfirstmiddle}{#1}
    \renewcommand{\ncsu@authorlast}{#2}
    \ncsu@testlastchar#1\relax
    \author{\ncsu@authorfirstmiddle{} \ncsu@authorlast}
}

%% Degree year - in case the degree year doesn't match the current year.
\newcommand{\ncsu@year}{\the\year}
\providecommand{\degreeyear}[1]{
    \renewcommand{\ncsu@year}{#1}
}

%% -------------------------------------------------------------------------- %%
%% Thesis Title
\newcommand{\ncsu@thesistitle}{My Thesis Title}
\providecommand{\thesistitle}[1]{\renewcommand{\ncsu@thesistitle}{#1}}

%% -------------------------------------------------------------------------- %%
%% ---------------------  Chapter Commands ---------------------------------- %%
%% -------------------------------------------------------------------------- %%

%% -------------------------------------------------------------------------- %%
%%  Redefine the \chapter* driver \@makeschapterhead
%% Center the chapter* head:
\AtBeginDocument{% at begin doc is necessary since fncychap
                 %  redefines \@makeschapterhead
    \renewcommand{\@makeschapterhead}[1]{%
        {\parindent0pt\centering\ncsu@headingformat#1\par\nobreak\vskip25pt}%
    }
}

%% -------------------------------------------------------------------------- %%
%% Abstract page commands
%%  - hyphnat package \nohyphens prevents title from being hyphenated.
%%  - \nolbreaks removes any line breaks in title (but you must use \linebreak
%%     instead of \\
\newcommand{\abstractname}{Abstract}
\newenvironment{abstract}{%
   \singlespacing
   \titlepage
   \chapter*{\abstractname}
   \MakeUppercase{\ncsu@authorlast{}, \ncsu@authorfirstmiddle{}}%
   \ifthenelse{\boolean{@ncsu@needsperiod}}{\@. }{~}%
    \nohyphens{\nolbreaks{\ncsu@thesistitle}.
   (Under the direction of
   \ifthenelse{\equal{\ncsu@chairtype}{chair}}%
              {\ncsu@chair}{\ncsu@cochairI\ and \ncsu@cochairII}.)}\\

   \ncsu@defaultspacing
   \thispagestyle{empty}
   \pagestyle{empty}
}%
{\endtitlepage \pagestyle{plain}}

%% -------------------------------------------------------------------------- %%
%% Abstract page commands
\newcommand{\makecopyrightpage}{
    \begin{titlepage}
    \centering
      \quad
      \vspace{0.35\textheight}

      \copyright{} Copyright \ncsu@year{} by \@author \\[1em]
      All Rights Reserved

      \vfill
    \end{titlepage}
}

%% -------------------------------------------------------------------------- %%
%% Title page commands

\newcommand{\ncsu@committeesize}{4}
\providecommand{\committeesize}[1]{\renewcommand{\ncsu@committeesize}{#1}}

%% To allow for Chair vs Co-Chairs:
\newcommand{\ncsu@chairtype}{}

%% For each committee member, we have their name and title.
%%  By default, their title is empty.  We use an optional argument
%%  to set the title of committee members or to set the chair's
%%  title to something other than `Chair of Advisory Committee'.
\newcommand{\ncsu@chair}{Advisor Q. Chair}
\newcommand{\ncsu@chairtitle}{Chair of Advisory Committee}
\providecommand{\chair}[2][]{
    \renewcommand{\ncsu@chair}{#2}
    % Checking if optional title argument is present.
    \ifthenelse{\equal{#1}{}}{}{\renewcommand{\ncsu@chairtitle}{#1}}
    % Declare that we have a chair:
    \renewcommand{\ncsu@chairtype}{chair}
}

\newcommand{\ncsu@cochairI}{Co-Chair 1 of Committee}
\newcommand{\ncsu@cochairItitle}{Co-Chair of Advisory Committee}
\providecommand{\cochairI}[2][]{
    \renewcommand{\ncsu@cochairI}{#2}
    \ifthenelse{\equal{#1}{}}{}{\renewcommand{\ncsu@cochairItitle}{#1}}
    % Declare that we have co-chairs.
    \renewcommand{\ncsu@chairtype}{cochair}
}

\newcommand{\ncsu@cochairII}{Co-Chair 2 of Committee}
\newcommand{\ncsu@cochairIItitle}{Co-Chair of Advisory Committee}
\providecommand{\cochairII}[2][]{
    \renewcommand{\ncsu@cochairII}{#2}
    \ifthenelse{\equal{#1}{}}{}{\renewcommand{\ncsu@cochairIItitle}{#1}}
    \renewcommand{\ncsu@chairtype}{cochair}
}

\newcommand{\ncsu@memberI}{Member 1 of Committee}
\newcommand{\ncsu@memberItitle}{}
\providecommand{\memberI}[2][]{
    \renewcommand{\ncsu@memberI}{#2}
    \ifthenelse{\equal{#1}{}}{}{\renewcommand{\ncsu@memberItitle}{#1}}
}

\newcommand{\ncsu@memberII}{Member 2 of Committee}
\newcommand{\ncsu@memberIItitle}{}
\providecommand{\memberII}[2][]{
    \renewcommand{\ncsu@memberII}{#2}
    \ifthenelse{\equal{#1}{}}{}{\renewcommand{\ncsu@memberIItitle}{#1}}
}

\newcommand{\ncsu@memberIII}{Member 3 of Committee}
\newcommand{\ncsu@memberIIItitle}{}
\providecommand{\memberIII}[2][]{
    \renewcommand{\ncsu@memberIII}{#2}
    \ifthenelse{\equal{#1}{}}{}{\renewcommand{\ncsu@memberIIItitle}{#1}}
}

\newcommand{\ncsu@memberIV}{Member 4 of Committee}
\newcommand{\ncsu@memberIVtitle}{}
\providecommand{\memberIV}[2][]{
    \renewcommand{\ncsu@memberIV}{#2}
    \ifthenelse{\equal{#1}{}}{}{\renewcommand{\ncsu@memberIVtitle}{#1}}
}

\newcommand{\ncsu@memberV}{Member 5 of Committee}
\newcommand{\ncsu@memberVtitle}{}
\providecommand{\memberV}[2][]{
    \renewcommand{\ncsu@memberV}{#2}
    \ifthenelse{\equal{#1}{}}{}{\renewcommand{\ncsu@memberVtitle}{#1}}
}

\newcommand{\ncsu@titlepageparagraph}{
 \ifthenelse{\equal{\ncsu@thesistype}{phd}}{
   A dissertation submitted to the Graduate Faculty of\\
   North Carolina State University\\
   in partial fulfillment of the\\
   requirements for the Degree of\\
   Doctor of Philosophy
 }{} % end ifthenels phd
 \ifthenelse{\equal{\ncsu@thesistype}{ms}}{
   A thesis submitted to the Graduate Faculty of\\
   North Carolina State University\\
   in partial fulfillment of the\\
   requirements for the Degree of\\
   Master of Science
 }{} % end ifthenels phd
}

\newcommand{\ncsu@program}{Degree Program}
\providecommand{\program}[1]{\renewcommand{\ncsu@program}{#1}}

\newcommand{\ncsu@citystate}{Raleigh, North Carolina}

\newcommand{\ncsu@signatureline}[2][]{
    \begin{minipage}[t]{0.4\textwidth}
    \centering
     \hrule
     \vspace{0.5ex}
     #2\\ #1
    \end{minipage}
}

\newcommand{\ncsu@committeesignatures}{
  \ifthenelse{\equal{\ncsu@chairtype}{chair}}{
    \ifthenelse{\ncsu@committeesize=3}{
      \begin{tabular}[t]{c@{\hspace{0.5in}}c}
         \ncsu@signatureline[\ncsu@memberItitle]{\ncsu@memberI} &
         \ncsu@signatureline[\ncsu@memberIItitle]{\ncsu@memberII} \\[6em]
         \multicolumn{2}{c}{\ncsu@signatureline[\ncsu@chairtitle]{\ncsu@chair}}
      \end{tabular}
    }{} % end ifthenelse = 3
    \ifthenelse{\ncsu@committeesize=4}{
      \begin{tabular}[t]{c@{\hspace{0.5in}}c}
         \ncsu@signatureline[\ncsu@memberItitle]{\ncsu@memberI} &
         \ncsu@signatureline[\ncsu@memberIItitle]{\ncsu@memberII} \\[6em]
         \ncsu@signatureline[\ncsu@memberIIItitle]{\ncsu@memberIII} &
         \ncsu@signatureline[\ncsu@chairtitle]{\ncsu@chair} \\
      \end{tabular}
    }{} % end ifthenelse = 4
    \ifthenelse{\ncsu@committeesize=5}{
      \begin{tabular}[t]{c@{\hspace{0.5in}}c}
         \ncsu@signatureline[\ncsu@memberItitle]{\ncsu@memberI} &
         \ncsu@signatureline[\ncsu@memberIItitle]{\ncsu@memberII} \\[5em]
         \ncsu@signatureline[\ncsu@memberIIItitle]{\ncsu@memberIII} &
         \ncsu@signatureline[\ncsu@memberIVtitle]{\ncsu@memberIV} \\[5em]
         \multicolumn{2}{c}{\ncsu@signatureline[\ncsu@chairtitle]{\ncsu@chair}}
      \end{tabular}
    }{} % end ifthenelse = 5
    \ifthenelse{\ncsu@committeesize=6}{
      \begin{tabular}[t]{c@{\hspace{0.5in}}c}
         \ncsu@signatureline[\ncsu@memberItitle]{\ncsu@memberI} &
         \ncsu@signatureline[\ncsu@memberIItitle]{\ncsu@memberII} \\[5em]
         \ncsu@signatureline[\ncsu@memberIIItitle]{\ncsu@memberIII} &
         \ncsu@signatureline[\ncsu@memberIVtitle]{\ncsu@memberIV} \\[5em]
         \ncsu@signatureline[\ncsu@memberVtitle]{\ncsu@memberV} &
         \ncsu@signatureline[\ncsu@chairtitle]{\ncsu@chair} \\
      \end{tabular}
    }{} % end ifthenelse = 6
  }{} % end ifthenelse type = chair
  \ifthenelse{\equal{\ncsu@chairtype}{cochair}}{
    % co-chairs; if number is odd, put co-chairs on top line.
    \ifthenelse{\ncsu@committeesize=3}{
      \begin{tabular}[t]{c@{\hspace{0.5in}}c}
         \ncsu@signatureline[\ncsu@cochairItitle]{\ncsu@cochairI} &
         \ncsu@signatureline[\ncsu@cochairIItitle]{\ncsu@cochairII} \\[6em]
         \multicolumn{2}{c}%
           {\ncsu@signatureline[\ncsu@memberItitle]{\ncsu@memberI}}
      \end{tabular}
    }{} % end ifthenelse = 3
    \ifthenelse{\ncsu@committeesize=4}{
      \begin{tabular}[t]{c@{\hspace{0.5in}}c}
         \ncsu@signatureline[\ncsu@memberItitle]{\ncsu@memberI} &
         \ncsu@signatureline[\ncsu@memberIItitle]{\ncsu@memberII} \\[6em]
         \ncsu@signatureline[\ncsu@cochairItitle]{\ncsu@cochairI} &
         \ncsu@signatureline[\ncsu@cochairIItitle]{\ncsu@cochairII}
      \end{tabular}
    }{} % end ifthenelse = 4
    \ifthenelse{\ncsu@committeesize=5}{
      \begin{tabular}[t]{c@{\hspace{0.5in}}c}
         \ncsu@signatureline[\ncsu@cochairItitle]{\ncsu@cochairI} &
         \ncsu@signatureline[\ncsu@cochairIItitle]{\ncsu@cochairII} \\[5em]
         \ncsu@signatureline[\ncsu@memberItitle]{\ncsu@memberII} &
         \ncsu@signatureline[\ncsu@memberIItitle]{\ncsu@memberIII} \\[5em]
         \multicolumn{2}{c}%
           {\ncsu@signatureline[\ncsu@memberIIItitle]{\ncsu@memberI}}
      \end{tabular}
    }{} % end ifthenelse = 5
    \ifthenelse{\ncsu@committeesize=6}{
      \begin{tabular}[t]{c@{\hspace{0.5in}}c}
         \ncsu@signatureline[\ncsu@memberItitle]{\ncsu@memberI} &
         \ncsu@signatureline[\ncsu@memberIItitle]{\ncsu@memberII} \\[5em]
         \ncsu@signatureline[\ncsu@memberIIItitle]{\ncsu@memberIII} &
         \ncsu@signatureline[\ncsu@memberIVtitle]{\ncsu@memberIV} \\[5em]
         \ncsu@signatureline[\ncsu@cochairItitle]{\ncsu@cochairI} &
         \ncsu@signatureline[\ncsu@cochairIItitle]{\ncsu@cochairII} \\
      \end{tabular}
    }{} % end ifthenelse = 6
  }{} % end ifthenelse type = cochair
}

\newcommand{\maketitlepage}{
     \thispagestyle{empty}
     \clearpage
     \singlespacing
      \begin{center}
            \vfill
        \ncsu@thesistitle
            \vfill
        by \\
        \@author
            \vfill
       \ncsu@titlepageparagraph
           \vfill
       \ncsu@program\\[2.5\ncsu@lineheight]
       \ncsu@citystate\\[1\ncsu@lineheight]
       \ncsu@year
           \vfill
       APPROVED BY:
           \vfill
       \ncsu@committeesignatures
      \end{center}
     \clearpage
     \ncsu@defaultspacing
}

%% -------------------------------------------------------------------------- %%
%% ------------ Dedication, biography, and acknowledgements ----------------- %%
%% -------------------------------------------------------------------------- %%
%%  Template
\newcommand{\ncsu@dedbioack}[1]{
    \chapter*{{\ncsu@headingformat{#1}}}
}

%%  Dedication
\newenvironment{dedication}
{\ncsu@dedbioack{Dedication}}
{\clearpage}

%%  Biography
\newenvironment{biography}
{\ncsu@dedbioack{Biography}}
{\clearpage}

%%  Acknowledgements
\newenvironment{acknowledgements}
{\ncsu@dedbioack{Acknowledgements}}
{\clearpage}

%% -------------------------------------------------------------------------- %%
%% -------------------------  Table of Contents ----------------------------- %%
%% -------------------------------------------------------------------------- %%
\renewcommand{\contentsname}{Table of Contents}
\settowidth{\cftchapnumwidth}{\textbf{Chapter 00}}
\renewcommand{\cftchappresnum}{Chapter }
\renewcommand{\cftchapleader}{\cftdotfill{\cftsecdotsep}}
\newcommand{\thesistableofcontents}{
    \singlespacing
    \tableofcontents
    \clearpage
    \ncsu@defaultspacing
}

\renewcommand{\listfigurename}{List of Figures}
\settowidth{\cftfignumwidth}{Figure 00.00}
\renewcommand{\cftfigpresnum}{Figure }
\newcommand{\thesislistoffigures}{
    \singlespacing
    \listoffigures
    \clearpage
    \ncsu@defaultspacing
}

\renewcommand{\listtablename}{List of Tables}
\settowidth{\cfttabnumwidth}{Table 00.00}
\renewcommand{\cfttabpresnum}{Table }
\newcommand{\thesislistoftables}{
    \singlespacing
    \listoftables
    \clearpage
    \ncsu@defaultspacing
}

%% -------------------------------------------------------------------------- %%
%% ------------------------- Landscape Pages  ------------------------------- %%
%% -------------------------------------------------------------------------- %%
%%  Loading pdflscape package for page rotation
\RequirePackage{pdflscape}
%%  Loading fancyhdr package to change page numbering
\RequirePackage{fancyhdr}
%%  Loading sideways for sideways environment
\RequirePackage{rotating}

\newlength{\portraittextheight}
\setlength{\portraittextheight}{\textheight}
\addtolength{\portraittextheight}{\footskip}
\newlength{\portraittextwidth}
\setlength{\portraittextwidth}{\textwidth}

\RequirePackage{afterpage}
\RequirePackage{changepage}

%% Rotate page and page number
\newenvironment{lscapenum}
{
%%   The standard landscape environment does not rotate the margins and
%%   page number.  However, we want the margins (mostly) rotated as well
%%   as the page number.  To change the margins, we use the \changepage
%%   command from the \changepage package.  Since the \textheight and
%%   \textwidth dimensions get switched, we need to reduce the rotated
%%   \textheight by \ncsu@footskip and increase the \textwidth by the
%%   same measure.
%%
%%   The syntax of the \changepage command is
%% \changepage{textheight}{textwidth}{evensidemargin}{oddsidemargin}{columnsep}
%%            {topmargin}{headheight}{headsep}{footskip}
   \clearpage
   \changepage{\ncsu@footskip}{-\ncsu@footskip}{}{}{}{}{}{}{}
   \begin{landscape}
     \pagestyle{fancy}
     \fancyhead{} % clear all header fields
     \fancyfoot{} % clear all footer fields
     \renewcommand{\headrulewidth}{0pt}
     \rfoot{
       %% Since the header and footer are unrotated, we must manually
       %% place the page number.  The picture environment is included in
       %% standard LaTeX.
       \setlength{\unitlength}{\portraittextheight}
       \begin{picture}(0,0)
           \put(0.055,0.557){\begin{sideways}\thepage\end{sideways}} % page number
       \end{picture}
     }

}
{
   \end{landscape}
   %% The \textheight is too large leaving landscape environment
   %%  since we invoked the \changepage.  We manually reduce it.
   \addtolength{\textheight}{-\ncsu@footskip}

   %% \textheight controls 3 other variables, which must be updated.
   %% See \endlandscape definition in lscape.sty
   \global\@colht=\textheight
   \global\vsize=\textheight
   \global\@colroom=\textheight

   \pagestyle{plain}
}

%% Rotate page but NOT page number
\newenvironment{lscapenotnum}
{
   \begin{landscape}
}
{
   \end{landscape}
}

%% -------------------------------------------------------------------------- %%
%% ---------------------------  Captions      ------------------------------- %%
%% -------------------------------------------------------------------------- %%
%%  Adding space to start of caption
%%   save
\RequirePackage{caption}
\DeclareCaptionFormat{plain}{#1:\hspace{0.8em}#3}

%% -------------------------------------------------------------------------- %%
%% -------------------------  References ------------------------------------ %%
%% -------------------------------------------------------------------------- %%
\renewcommand{\bibname}{References}
%% natbib redefines \thebibliography, so we add our own requirements
%% at \begin{document} time.
\AtBeginDocument{
    \let\ncsu@thebibliography\thebibliography
    \let\endncsu@thebibliography\endthebibliography
    \renewenvironment{thebibliography}[1]
    {
        \singlespacing
        \begin{ncsu@thebibliography}{#1}
    }
    {
        \end{ncsu@thebibliography}
        \ncsu@defaultspacing
    }
}

%% -------------------------------------------------------------------------- %%
%% ------------------------- The Appendices  -------------------------------- %%
%% -------------------------------------------------------------------------- %%

\newlength{\appendixwidth}
\settowidth{\appendixwidth}{Appendix A}
\addtolength{\appendixwidth}{1em}
\providecommand{\appendixheader}[1]{
    \renewcommand{\ncsu@appendixheader}{#1}
}

%% Count the total number of appendices
\RequirePackage{totcount}
\regtotcounter{chapter} % chapter counter is used for appendices

\AtBeginDocument{%
%% Since ETD requirements are different than LaTeX's normal formatting of
%% the Table of Content's Appendix listings,
%% we change the indentation of chapters, sections, and subsections in
%% the appendix part of the table of contents, but no deeper divisions,
%% so we must cap the tocdepth at 2:
\newcounter{ncsu@appendixtocdepth}
\ifthenelse{\c@tocdepth > 2}
{\setcounter{ncsu@appendixtocdepth}{2}}
{\setcounter{ncsu@appendixtocdepth}{\c@tocdepth}}

\let\ncsu@appendix\appendix
\renewcommand{\appendix}{
    \ifthenelse{\isundefined{\ncsu@appendixheader}}
    % Appendix title page heading not defined, define it
    {
        \ifthenelse{\totvalue{chapter} = 1 }%
        %% number of appendices = 1
        {\newcommand{\ncsu@appendixheader}{Appendix}}
        %% number of appendices > 1
        {\newcommand{\ncsu@appendixheader}{Appendices}}
    }
    % Appendix title page heading already defined, do nothing
    {}

    \chapter*{\ncsu@appendixheader} %% insert 'Appendix' titlepage
    \addcontentsline{toc}{chapter}{\ncsu@appendixheader}
    %% Adding "Appendix" before letter in TOC
    %%  We must change the width of the box which the appendix letter is set in
    %%  and add the "Appendix " before the letter.  See tocloft documentation.
    \addtocontents{toc}{
        \protect\renewcommand{\protect\cftchappresnum}{Appendix }
        \protect\renewcommand{\protect\cftchapnumwidth}{\protect\appendixwidth}
        \protect\renewcommand{\protect\cftchapfont}{\protect\cftsecfont}
        \protect\setlength{\protect\cftbeforechapskip}{\protect\cftbeforesecskip}
        \protect\setlength{\protect\cftchapindent}{\protect\cftsecindent}
        \protect\renewcommand{\protect\cftchappagefont}{\protect\cftchapfont}
        \protect\setlength{\protect\cftsecindent}{\protect\cftsubsecindent}
        \protect\addtolength{\protect\cftsubsecindent}{\cftsecnumwidth}
        \protect\setcounter{tocdepth}{\c@ncsu@appendixtocdepth}
    }
    \clearpage
    \ncsu@appendix %% call internal \appendix macro
} % \renewcommand{appendix}
} % \AtBeginDocument

\ncsu@defaultspacing
%</class>
%\Finale
